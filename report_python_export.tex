\documentclass[a4paper,12pt]{article}
\usepackage{tcolorbox}  % for colored boxes
\usepackage{xcolor}  % to properly define colors 
\usepackage{titling}  % to help with title spacing
\usepackage{array}
\usepackage{makecell}
\usepackage{longtable}

\renewcommand\theadalign{bc}
\renewcommand\theadfont{\bfseries}
\renewcommand\theadgape{\Gape[4pt]}
\renewcommand\cellgape{\Gape[4pt]}

\definecolor{framecolor}{RGB}{255, 255, 255}  % Rule colour
\definecolor{fillcolor0}{RGB}{255,127.5,127.5}
\definecolor{fillcolor1}{RGB}{255,127.5,127.5}
\definecolor{fillcolor2}{RGB}{255,163.59267305957067,127.5}

\makeatletter
\newcommand{\mybox}[2]{
 	\setbox0=\hbox{#1}
 	\setlength{\@tempdima}{\dimexpr\wd0+13pt}
 		\begin{tcolorbox}[colframe=framecolor,colback={#2}, boxrule=0.5pt,arc=4pt, left=6pt,right=6pt,top=6pt,bottom=6pt,boxsep=0pt,width=\@tempdima]
   		#1
 		\end{tcolorbox}
}
\makeatother

\setlength{\droptitle}{-10em}   % This is your set screw
\title{US Recession Forecast: \today}
\author{Danny Cohen, Corporate Intelligence Analyst Intern}
\date{}

 \begin{document}

	\maketitle

	\section{Recession Probabilities} 
		
		\subsection{Three-month Predictions}

			\begin{tabular*}{\textwidth}{c @{\extracolsep{\fill}} ccccc}
				\\
				\textit{December  2019} & &\textit{January  2020} & & \textit{February  2020} \\ 
				\\
				\mybox{\huge{22.64\%}}{fillcolor0} & $\Longrightarrow$ & \mybox{\huge{24.56\%}}{fillcolor1} &
					 $\Longrightarrow$ & \mybox{\huge{25.38\%}}{fillcolor2} \\
				$\uparrow$ 20.38 & & $\uparrow$ 22.26 & & $\uparrow$ 0 \\
			\end{tabular*}

		\subsection{Interpreting the Data}

			Do not treat these numbers as mere raw percentages. Based on previous analysis, values above 5\% indicate an elevated risk of recession; values above 10\% indicate a significant risk of recession; and values above 20\% indicate near-certainty of recession. \\ \\
			The algorithm is designed to underestimate recession risk, meaning high recession probabilities should not be taken lightly.  \\ \\ 
			\textit{Disclaimer}: these algorithms were trained exclusively on past recessions. Most downturns share macroeconomic characteristics, which makes prediction a reasonable task. However, any future recession would likely go unnoticed if its causes were novel and previously unseen, at which point prediction would be akin to gambling. 

	\pagebreak

	\section{Independent Variables \& Sources}
		\begin{longtable}{p{.40\textwidth} p{.60\textwidth}}
			\hline
			\thead{\textit{Key}} & \thead{\textit{Source}} \\
			\hline
			M1 & http://www.federalreserve.gov/releases/h6/\\
DTWEXBGS & http://www.federalreserve.gov/releases/h10/\\
UMCSENT & http://www.sca.isr.umich.edu/\\
PERMIT & http://www.census.gov/construction/nrc/\\
ASPUS & http://www.census.gov/construction/nrs/\\
HSN1F & http://www.census.gov/construction/nrs/\\
AWHAEMAN & http://www.bls.gov/ces/\\
TCU & http://www.federalreserve.gov/releases/g17/\\
RSXFS & http://www.census.gov/retail/\\
CPIAUCSL & http://www.bls.gov/cpi/\\
CPALTT01USM657N & http://www.oecd-ilibrary.org/economics/data/main-economic-indicators/main-economic-indicators-complete-database\_data-00052-en\\
DTWEXB & http://www.federalreserve.gov/releases/h10/\\
NASDAQCOM & http://www.nasdaq.com/\\
BOGZ1LM155035015A & http://www.federalreserve.gov/releases/z1/\\
AMTMNO & http://www.census.gov/indicator/www/m3/\\
MANEMP & http://www.bls.gov/ces/\\
INDPRO & http://www.federalreserve.gov/releases/g17/\\
DJIA & http://www.djaverages.com/\\
AWHMAN & http://www.bls.gov/ces/\\
NEWORDER & http://www.census.gov/indicator/www/m3/\\
IPMAN & http://www.federalreserve.gov/releases/g17/\\
A36SNO & http://www.census.gov/indicator/www/m3/\\
MCUMFN & http://www.federalreserve.gov/releases/g17/\\
TOTBUSMPCIMSA & http://www.census.gov/mtis/www/mtis.html\\
M2 & http://www.federalreserve.gov/releases/h6/\\
MSPUS & http://www.census.gov/construction/nrs/\\
ESALEUSQ176N & http://www.census.gov/housing/hvs/\\
ACDGNO & http://www.census.gov/indicator/www/m3/\\
IPDCONGD & http://www.federalreserve.gov/releases/g17/\\
AMDMUO & http://www.census.gov/indicator/www/m3/\\
AMTMUO & http://www.census.gov/indicator/www/m3/\\
BUSINV & http://www.census.gov/mtis/www/mtis.html\\
T10YFF & \\
CPROFIT & https://www.bea.gov/data/gdp/gross-domestic-product\\
ICSA & http://www.dol.gov/ui/data.pdf\\
DTWEXM & http://www.federalreserve.gov/releases/h10/\\
CSCICP03USM665S & http://www.oecd-ilibrary.org/economics/data/main-economic-indicators/main-economic-indicators-complete-database\_data-00052-en\\
HOUST & http://www.census.gov/construction/nrc/\\
DGORDER & http://www.census.gov/indicator/www/m3/\\
AMDMUS & http://www.census.gov/indicator/www/m3/\\
ACOGNO & http://www.census.gov/indicator/www/m3/\\
TOTALSA & https://www.bea.gov/national/xls/gap\_hist.xlsx\\
ECOMSA & http://www.census.gov/mrts/www/ecomm.html\\
UNRATE & http://www.bls.gov/ces/\\
IC4WSA & http://www.dol.gov/ui/data.pdf\\
SP500 & https://us.spindices.com/indices/equity/sp-500\\
MICH & http://www.sca.isr.umich.edu/\\
BSCICP03USM665S & http://www.oecd-ilibrary.org/economics/data/main-economic-indicators/main-economic-indicators-complete-database\_data-00052-en\\
ETOTALUSQ176N & http://www.census.gov/housing/hvs/\\
LCEAMN01USM659S & http://www.oecd-ilibrary.org/economics/data/main-economic-indicators/main-economic-indicators-complete-database\_data-00052-en\\
PCDG & https://www.bea.gov/data/gdp/gross-domestic-product\\
MARTSMPCSM44000USS & http://www.census.gov/retail/\\
RETAILMPCSMSA & http://www.census.gov/mtis/www/mtis.html\\
CPALTT01USQ657N & http://www.oecd-ilibrary.org/economics/data/main-economic-indicators/main-economic-indicators-complete-database\_data-00052-en\\
			\hline
		\caption{Independent variables keys and their associated sources.}
		\label{tab:myfirstlongtable}
		\end{longtable}

	\pagebreak

	\section{Process}
		\begin{enumerate}
			\item Retrieve data for each of the 58 independent variables from FRED server.
			\item Clean, interpolate, and aggregate these data.
			\item Retrieve dependent variables from FRED, and offset by desired number of months.
			\item Combine independent and dependent variables and merge with VMware internal data.
			\item Upload tables to \textit{sse\_ccmi} schema in PostgreSQL data warehouse: 
				\begin{itemize}
					\item \textit{dc\_model\_aggregate}: independent variables
					\item \textit{dc\_model\_merged}: independent and dependent variables
					\item \textit{dc\_model\_sources}: independent variables and their sources
					\item \textit{dc\_model\_vmware\_sales\_2010\_onward}: VMware sales 2010-present
					\item \textit{dc\_model\_fred\_vmware\_combined}: independent and dependent variables, VMware sales data
					\item \textit{dc\_model\_complete}: independent and dependent variables, VMware sales data, dependent-variable 
						predictions
				\end{itemize}
			\item Read data into Python's TensorFlow machine learning framework.
			\item Divide dataset into five segments, one for each recession in the past 48 years.
			\item For each segment: train three new boosted regression tree algorithms on the other four segments and pick the most accurate iteration.
			\item Use five algorithms (best one from each segment) to predict state of macroeconomy.
			\item Export report as PDF.
		\end{enumerate}

 \end{document}